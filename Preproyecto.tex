\documentclass[english]{article}
\usepackage[utf8]{inputenc} 
\usepackage[T1]{fontenc}
\usepackage{mathpazo,amsmath,amssymb,subcaption,graphicx,amsthm} 
\usepackage[backend=bibtex,style=numeric-comp,natbib=true]{biblatex} 
\usepackage{xcolor}
\addbibresource{references.bib}

\title{Preproject}
\author{Eliana Tolosa Villarreal and Jerónimo Valencia}
\date{October 2020}

\newtheorem{theorem}{Theorem}[section]
\newtheorem{corollary}{Corollary}[theorem]
\newtheorem{lemma}[theorem]{Lemma}
\theoremstyle{definition}
\newtheorem{definition}{Definition}[section]
\theoremstyle{definition}
\newtheorem{proposition}{Proposition}[section]
\theoremstyle{definition}
\newtheorem{example}{Example}[section]
\theoremstyle{remark}
\newtheorem*{remark}{Remark}

\begin{document}
\maketitle

Given a graph $G$ with $n$ vertices, we can associate a zonotope $Z_G$ in $\mathbb{R}^n$ given by the Minkowski sum of segments indexed by the edges of such graph. Explicitly, $$Z_G = \sum_{\{ij\}\in\text{E}(G)}\left[\vec{e}_i,\vec{e}_j\right] = \sum_{\{ij\}\in\text{E}(G)}\left[0,\vec{e}_i-\vec{e}_j\right]$$
where $\left\{\vec{e}_i \ | \ 1\leq i\leq n \right\}$ is the canonical basis for $\mathbb{R}^n$. For these polytopes the following proposition describes a combinatorial formula for computing its volume. 

\begin{proposition}
\cite[Proposition 2.4]{Postnikov-PAB} For a connected graph $G$ on $n$ vertices, the volume of $Z_G$ equals the number of spanning trees of $G$. 
\end{proposition}

The proof of this proposition relies on a particular subdivision, which can be done to any zonotope, into smaller zonotopes \cite{beck2008computing}. 

Now, if we consider a simplicial complex $\Gamma$ such that $|\Gamma^{(0)}|=n$, a polytope $Z_\Gamma$ in $\mathbb{R}^n$ analogous to the graphical zonotope can be considered as follows. If $F\in\Gamma$ is a face of dimension $k$, we define $\Delta_F \subset \mathbb{R}^n$ as the convex hull of vectors indexed by elements of $F$. Then, $$Z_\Gamma := \sum_{F\in\Gamma}\Delta_F$$

This construction generalizes the above polytope because in the case $\Gamma$ is a graph, $Z_\Gamma$ is a translate of the graphical zonotope by a vector $\vec{v} = \sum_{i\in \Gamma^{(0)}}\vec{e}_i$. 

The purpose of our project is to try to compute the volume of these simplicial complex polytopes combinatorially and therefore to generalize the proposition mentioned above for graphical zonotopes.

In order to approach this computation we will study subdivisions and zonotopes (from the cited references). Given the case that we do not get to a general result, the project will turn into a survey on graphical zonotopes, in which we will show some computations of volumes as examples. In that case, we could also focus in a smaller family of polytopes to which a volume could be computed.

Each week we will set tasks to do. We will have a weekly meeting to discuss our progress. We will write the paper collaboratively, having in mind our individual strengths.


\nocite{*}

\printbibliography

\end{document}
